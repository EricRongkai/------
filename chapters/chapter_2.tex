\chapter{研究理论基础}

\section{闭环人机交互过程的不确定性分析}
人-机器人交互问题可以被认为是一个协商性的协同控制过程,在此过程中使用者与系统共同沟通意图。因此,每一种类型的交互过程都有某种形式的不确定性,通常我们的目标都是引入更多的传感器信息去试图设计一个更精准的方法去推断使用者的意图从而忽略其中不确定因素的存在。我们认为在人机交互过程中的不确定性本身是一种信息,应该需要被正确的认识和对待,并且必须表示和处理所存在的不确定性,而不是盲目地过滤或忽略掉它。人机交互过程的不确定性来源繁杂,涉及人的行为、决策、机器的感知和决策、协同任务的目标设定以及交互方式等多个方面,大致有以下几种类型:(1)人的行为不确定性:人的行为受到情绪、健康状况、认知能力等因素的影响,导致其在与机器进行协同工作时的表现不确定。(2)人类决策的不确定性:人类在面临选择时,由于信息不完全、知识有限、环境复杂等因素导致无法确定最佳选择的情况。(3)机器的感知不确定性:机器在与人进行协同工作时,需要对人的行为和环境进行感知和理解。然而,由于感知技术的限制,机器对于人的行为和环境的理解往往存在不确定性。(3)机器决策的不确定性:由于机器学习算法的限制或数据的不完整性、不准确性等因素导致机器无法确定最佳决策的情况。(4)协同任务的目标不确定性:协同任务的目标通常由人和机器共同制定,但由于人和机器的意见、知识和经验的不同,目标的制定存在一定的不确定性。(5)交互方式的不确定性:人机协同中的交互方式往往是通过交互界面、语言或手势等方式进行的,但这些交互方式存在一定的不确定性。

已有研究对人-机器人交互在多智能体框架中进行建模分析,其中机器人和人被都被视为智能代理\cite{liuDesigningRobotBehavior}。因此,我们首先从具有较好理论研究基础的智能机器人行为设计出发,对机器智能的设计实现过程进行分析。为了尽可能地适用于更多的交互场景和传感器使用,在此基础上,通过将人类交互行为视为一个连续的闭环控制过程,我们进一步分析了交互过程中的不确定性因素来源。

\subsection{机器人行为建模}
为了生成所期望的机器人行为,从优化控制的角度我们需要从三个方面进行设计\cite{liuDesigningRobotBehavior}:(1)以内部成本的形式向机器人提供有关任务要求的正确知识,并且向机器人提供关于外部环境的内部动力学模型;(2)设计一个的正确的决策模型,使得机器人可以将内部的知识表征转化为想要的动作;(3)设计一个学习方法来更新机器人的内部知识和决策模型,使得机器人可以在不同的环境中的动作决策具有泛化性。在图\ref{fig:2-1}中给出了机器人行为建模设计的框图,其中知识表征、决策模型以及学习算法为三个主要部分。机器人通过从人机混合的外部环境中获取环境数据$s$并根据其决策模型$d$和内部的损失函数以及内部的动力学模型约束$M$产生动作$a$。由于环境改变或所设计的知识表征无法满足当前场景的处理,因此需要学习来自环境的数据以更新内部的知识表征和决策模型。

机器人的行为系统开发主要分为三个阶段:设计阶段、训练阶段和执行阶段,其中设计和训练通常是离线完成的,而执行是在线进行的。在训练阶段,机器人可以从人类经验或来自于人类的示教演示中学习知识。其中从人类的示教中学到的知识与人类设计的知识之间的区别在于,前者仅需要提供数据而不需要人类对知识的表征有数学或定量的表示。因为在大多数情况下,知识是抽象的且对人类来说是不直观的,我们往往无法显式地描述一个技能或知识,因此难以获得可靠的知识数学表征(例如,人类直接绘制出一个轨迹比设计出该轨迹的数学函数表达式更容易)。在执行阶段,机器人通过执行特定的任务并与人类进行交互,并且可以通过在线学习来更新这些知识或逻辑。然而,由于计算能力的限制,在线学习通常仅限于小规模的参数自适应。如过需要从头开始学习一种新技能,一般只能通过在训练阶段的离线学习来完成。其中训练阶段和执行阶段可以在一个学习系统中迭代地执行,当外部环境或任务较为简单时,机器人的行为也可以不经过训练直接从设计阶段到执行阶段。

\begin{figure}[h]
    \centering
    \includegraphics[width=0.6\textwidth]{2-fig-1.pdf}
    \caption{机器人行为建模设计结构框图}
    \label{fig:2-1}
\end{figure}
知识表征是一个机器人行为系统的核心,但是现在对于知识中先天设计与后天学习各自所占的比重应当是多少仍然需要被研究。而行为系统中的另外两个组件,决策模型和学习过程的本质是一种算法,因此通常是由人类进行设计的。决策模型的设计通常由三种方式,其中前两种方法是基于模型的,需要给出显式的$J$和$M$:(1)在设计阶段,明确地求解以$M$为动力学约束的优化控制问题$d(s)=\min_u J(u,s)$进而得到一个精确的策略,由于内部代价是非凸的,所以决策函数$d$可能是不连续的;(2)在执行阶段在线求解优化控制问题,由于$J$的非凸性,在线计算的控制输入u可能只是一个局部最优;(3)在训练阶段使用无模型的参数函数(例如神经网络)来近似该策略。综合来看,决策模型的设计通常由以下四类:

\begin{itemize}
\item 由设计者指定内部成本和内部模型以及显式的决策函数表达,并明确地解决优化问题,不需要任何学习过程。代表性的方法是:(1)经典控制方法;(2)马尔可夫决策过程(MDP);(3)经典模型预测控制(MPC)方法。
\item 由设计者指定内部成本以及显式的决策函数表达,基于学习对内部模型进行辨识。代表性的方法是:(1)经典的自适应控制;(2)自适应MPC控制器。该类型方法的优点是,它可以对时变的外部环境进行处理,特别适用于人在环中的场景。
\item 由设计师明确设计决策算法和相应的学习方法。知识在训练阶段通过试错或专家演示来获得的。代表性的方法是:(1)基于模型的强化学习;(2)逆强化学习,如学徒学习。这种方法的优点是,在设计阶段不再需要对任务和环境进行数学建模。
\item 由设计者明确地设计学习方法,并使用一个参数函数(例如神经网络)来近似决策模型。机器人将在训练阶段获得关于环境的知识(例如,网络中的参数),因此在这类方法中知识不是明确地学习的,而是隐式地编码在网络中。代表性的方法是:(1)无模型强化学习,如深度强化学习;(2)模仿学习。该方法通常适用于任务和环境极其难以建模,或者状态空间太大以及对实时计算要求较高的场景中。
\end{itemize}

\subsection{闭环人机交互过程}

\section{人机共享自主}

\section{机器人轨迹模仿学习}