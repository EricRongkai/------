% !TeX root = ../main.tex

\ustcsetup{
  keywords  = {康复机器人, 辅助机器人, 人机交互, 共享自主},
  keywords* = {Rehabilitation Robot, Assistive Robot, Human-Robot Interaction, Shared Autonomy},
}

\begin{abstract}
人口老龄化已经成为我国近年来持续面临的主要挑战之一,康复辅助机器人有望在未来发挥重要作用,并为应对人口老龄化问题提供有力的支持。在康复辅助机器人的人机交互系统设计中,需要综合考虑安全性和个体适应性是其区别于其他种类机器人的主要特征。然而,这两者往往是相悖的,过分强调安全排除一切可能造成风险的不确定性因素会导致机器人倾向于保守进而降低使用者体验。另一方面,人类行为虽然在许多方面明显有规则约束并且可预测,但它也始终是可变的、个体的,甚至是随机的,将更多使用者的自主行为引入交互系统中则会导致系统存在不可控的风险。通过有效地应对人机交互中的不确定性,康复辅助机器人能够根据患者的需求和反馈进行个性化调整,从而实现康复辅助机器人在个体适应性和安全性之间的最佳平衡。目前,共享自主方法已被证实为应对上述问题的一种有效的解决方案。通过对人类行为进行概率建模,共享自主系统可以有效地将人机交互过程中的不确定性因素利用起来,进而自适应地调整系统的行为和响应,对于提高辅助系统性能和用户体验至关重要。本文围绕共享自主方法,从机器人行为建模方法出发,对闭环人机交互过程中存在的不确定性因素进行了分析。通过搭建实验平台,深入探讨和研究了其在以下三个典型场景中的应用潜力与挑战:(1)定制化人机交互接口在移动辅助机器人操控中的自适应交互指令解码。基于一个自主开发的新型柔性可穿戴人机交互接口,通过显式表达人机交互过程中的不确定性,分别开发了用于交互指令生成的两种数据解码方法:确定性直接数据映射解码方法,以及不确定性意图推理解码方法。此外,我们将用户在使用确定性数据映射解码方式操控光标的先验表现集成到一个共享自主系统的非线性仲裁函数中,提出了一种自适应切换数据解码方法。实验证明,其不仅提高了使用者操控命令生成的准确性,同时保证了数据解码方法的动态性能以适应不同任务的要求;(2)站立辅助机器人中被辅助对象的运动交互意图自适应。针对人类在完成坐立运动时的移动速度不确定性,我们通过运动捕捉设备采集真实场景下的人体坐立离线运动数据,建立了一个由概率化离散动态运动基元表征的先验技能库。根据当前对被辅助对象下肢运动的部分观测,通过将运动时间估计看做一个系统参数辨识问题,基于期望最大化算法迭代优化计算模型时间缩放参数,实现了连续的人体坐立运动时间估计。所提出方法可以通过一个估计置信度水平量化指标嵌入一个共享自主系统中,进而实现辅助机器人的在线运动轨迹的稳定优化;(3)用于偏态步态康复的主动式膝关节辅助机器人自适应运动轨迹规划方法。围绕一个用于偏瘫患者步态对称性康复训练的主动式膝关节矫形器原型样机,我们详细地介绍了一种通过学习偏瘫患者健侧步态特征并用于患侧执行器轨迹规划的在线对称步态轨迹生成方法。该方法通过融合节律型动态运动基元与一个耦合自适应非线性频率振荡器,实现了步态运动周期轨迹的在线的编码/解码以及步态相位的自适应延迟。此外,通过离线采集健康人群的步行示教数据,设计了一个带有先验步态技能库的共享自主系统。在非结构化环境下,其通过自适应地分析和仲裁来自健侧下肢的实时用户自主输入,从而最大程度地减轻输入不确定性对步态运动轨迹生成过程的影响。
\end{abstract}

\begin{abstract*}
Population aging has become one of the major challenges that our country has been facing in recent years, and rehabilitation assistive robots are expected to play an important role in the future, providing strong support for addressing the issue of population aging. In the design of human-computer interaction systems for rehabilitation assistive robots, it is necessary to comprehensively consider safety and individual adaptability, which are the main characteristics that distinguish them from other types of robots. However, these two factors are often contradictory. Overemphasizing safety by eliminating all possible uncertain factors that may cause risks can lead to robots being overly conservative and thus reducing user experience. On the other hand, although human behavior is clearly rule-bound and predictable in many ways, it is also always variable, individual, and even random. Introducing more user autonomy into the interaction system can lead to uncontrollable risks in the system. By effectively dealing with uncertainty in human-computer interaction, rehabilitation assistive robots can be personalized according to the needs and feedback of patients, thereby achieving the optimal balance between individual adaptability and safety. Currently, the shared autonomy approach has been proven to be an effective solution to the above problems. By probabilistically modeling human behavior, shared autonomy systems can effectively utilize uncertain factors in the human-computer interaction process, and then adaptively adjust the system's behavior and response, which is crucial for improving the performance of assistive systems and user experience. This article focuses on the shared autonomy approach, starting from the robot behavior modeling method, and analyzes the uncertain factors in the closed-loop human-computer interaction process. By building an experimental platform, the application potential and challenges in the following three typical scenarios are deeply explored and studied. (1) Adaptive interaction command decoding in customized human-computer interaction interfaces for mobile assistive robot control. Based on a newly developed flexible wearable human-computer interaction interface, by explicitly expressing the uncertainty in the human-computer interaction process, two data decoding methods for generating interaction commands have been developed: a deterministic direct data mapping decoding method, and an uncertain intention inference decoding method. Furthermore, we integrated the prior performance of users controlling the cursor using the deterministic data mapping decoding method into the nonlinear arbitration function of a shared autonomy system, proposing an adaptive switching data decoding method. Experiments have proven that this method not only improves the accuracy of user-generated control commands but also ensures the dynamic performance of the data decoding method to adapt to different task requirements; (2) Adaptive motion interaction intention of the assisted object in standing assistive robots. Addressing the uncertainty in human movement speed during sit-to-stand movements, we collected offline human sit-to-stand motion data in real scenes using motion capture equipment and established a prior skill library characterized by probabilistic discrete dynamic movement primitives. Based on the partial observation of the lower limb movement of the assisted object, by treating motion time estimation as a system parameter identification problem, we iteratively optimized the model time scaling parameters based on the Expectation-Maximization algorithm, achieving continuous human sit-to-stand motion time estimation. The proposed method can be embedded into a shared autonomy system through an estimated confidence level quantitative index, thereby realizing stable optimization of the assistive robot's online motion trajectory; (3) Adaptive motion trajectory planning method for active knee assistive robots used in asymmetric gait rehabilitation. Focusing on a prototype of an active knee orthosis for gait symmetry rehabilitation training in hemiplegic patients, we proposed an online symmetrical gait trajectory generation method that learns from the healthy side gait characteristics of hemiplegic patients and applies them to the trajectory planning of the affected side actuator. This method achieves online encoding/decoding of gait movement cycle trajectories and adaptive delay of gait phases by integrating rhythmic dynamic movement primitives with a coupled adaptive nonlinear frequency oscillator. Additionally, by collecting offline walking demonstration data from healthy individuals, a shared autonomy system with a prior gait skill library was designed. In unstructured environments, it adaptively analyzes and arbitrates real-time user autonomous inputs from the healthy lower limb, thereby minimizing the impact of input uncertainty on the gait movement trajectory generation process.
\end{abstract*}
