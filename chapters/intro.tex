% !TeX root = ../main.tex

\chapter{绪论}

\section{研究背景与意义}
人口老龄化问题成为我国近年来持续面临的挑战,据《2021年度国家老龄事业发展公报》\cite{2021NianDuGuoJiaLaoLingShiYeFaZhanGongBao}的数据指出,截至2021年末,全国60周岁及以上老年人口26736万人,占总人口的18.9\%;全国65周岁及以上老年人口20056万人,占总人口的14.2\%。全国65周岁及以上老年人口抚养比20.8\%。其中《中国养老服务蓝皮书(2012—2021)》\cite{YangZhongGuoYangLaoFuWuLanPiShu201220212022}中的数据显示,2015年我国失能、部分失能老年人达到4063万人,占老年人口的18.3\%。与此同时,到2025年我国失能总人口预计将上升到7279.22万人,2030年将达1亿人,空巢和独居老年人已达到 1.18 亿。由于祖辈和子代两地分居,子代对祖辈的照顾多来自于经济支持,而缺少对于生活照护、情感陪伴等家庭养老支持,其严重弱化了独居与失能、半失能老年人的生活状况。面向助老助残领域的康复辅助机器人研究对于减轻社会养老负担,提升失能老年人生活独立性和自信心具有重要意义。

康复辅助机器人一般用于辅助或扩展人类的运动和/或认知能力,主要面向对象为:体弱的老人、截肢者和其他患有脊髓损伤、中风等疾病的人群。 近年来,我国先后颁布了《中国制造2025》\cite{ZhongGuoZhiZao2025}、《“健康中国2030”规划纲要》\cite{LiuYangJianKangZhongGuo2030GuiHuaGangYao}等文件,重点支持康复辅助机器人产业发展。辅助机器人一般需要其拥有智能化和鲁棒性以维持系统的安全高效性和灵活性,通过集成在线信息处理机制,机电一体化和先进的人机交互接口与人进行物理或者其他感官接触。尽管近年来机器人技术以及模式识别技术取得了巨大的进展,在辅助机器人领域,我们仍远未为机器人提供完全的自主权,从而使它们能够在动态变化的环境中能够处理更多的情况。

近年来,以机器学习和深度学习为代表的机器智能技术的快速发展并取得显著成果。然而,随着数据量增加,``智能提升''逐渐减弱。与此同时,人工智能系统与人类之间的互动关系变得更加广泛和复杂\cite{ZhaoQianTanKongZhiZhongDeGongXiangXinXiHeGongXiangZiZhu2021}。相较于人类智能,机器智能通常根据当前对外部环境的感知给出确定性的决策,灵活程度低且容易受到感知系统输入不确定性的影响导致决策失败。然而,机器决策在动态变化的环境下往往可以产生高维度的操控指令并作出最优的决策,进而保证系统的整体安全高效。然而,在大多数辅助机器人应用中,通常都由人类操作者操作或监督机器人。在康复辅助机器人系统中,使用者一般都存在显著认知和运动障碍,通常仅能使用有限离散的交互接口导致基于交互界面的人类决策输入存在更高程度的交互不确定性,容易造成意外情况出现。相较于机器智能,人类操作者能够提供优越的情境感知、逻辑和解决问题的能力,但这也提出了一个核心问题:在需要保证安全可靠性的人-辅助机器人交互系统中,我们应当怎样处理用户的交互输入的不确定性。

在物理人-机器人交互领域,机器自主是一种手段,而不是目标,自主级别因应用领域的不同存在差异。与智能控制器共享机器人系统的控制,可以让人类在执行任务时减少认知和身体上的工作量,在另一方面机器智能又能够在一定程度上纠正使用者的错误输入。在``通用人工智能''时代真正到来之前,人机混合将长期存在于辅助机器人系统中\cite{ZhangMianXiangRenJiXuGuanJueCeDeHunHeZhiNengFangFaYanJiu2021}。因此,如何在康复辅助机器人中动态处理人类交互输入不确定性,对于提高康复辅助机器人系统的鲁棒性和安全性,减轻使用者负担,促进相关技术应用落地具有重要意义。

\section{康复辅助机器人研究现状}
康复辅助机器人的研究涉及机械,信息,控制,计算机,医学等多个学科,是当前机器人研究领域的热点。目前的辅助设备如电动轮椅、辅助机械臂、上肢或下肢假肢和主动矫形器,对于帮助那些有严重运动障碍的人改善独立生活能力、减轻家庭照护负担至关重要。
\subsection{主动矫形器和智能假肢}
主动矫形器与与智能假肢系统经过数代发展,涌现出多种拥有不同驱动方式和控制方法的系统。其中控制系统是其核心部分,需要系统能够实时感知用户的运动意图和姿态,并根据这些信息控制机器人的动作,常用的控制方法包括基于模型的控制、基于传感器的控制和混合控制等。目前,相关研究更加强调在多学科领域的整合,包括生物力学、机器人工程、神经科学等多个学科的交叉,推动了该领域的不断创新。相较于外骨骼机器人,主动式矫形器和智能假肢主要用于帮助患者恢复行走能力,改善步态,减轻关节负担等,通常采用被动式或半主动式驱动方式,即根据患者的步态自动调整其支撑力和运动模式。而下肢外骨骼机器人则采用全主动式驱动方式,可以根据患者的需求和外部指令进行精确的运动控制,并用于康复训练、提高运动能力、增强体力等。

相对于需要携带自己的电池和执行器的完整外骨骼来说,主动矫形器普遍都有更低质量和转动惯量,因此驱动器的输出功率可以更多用于为用户提供辅助而不是补偿自身重量\cite{collinsReducingEnergyCost2015,zhangHumanintheloopOptimizationExoskeleton2017a}。其一般多为单关节结构,在保留使用者一部分自由活动能力的前提下通过驱动器辅助特定关节来完成运动动作,其可分为髋关节、膝关节或踝关节辅助系统\cite{malcolmExperimentalStudyRole2009}。在步行中,膝关节主要是一个自由阻尼关节,在摆动阶段几乎处于锁定状态,而在支撑阶段则与之相反;髋关节和踝关节主要与摆动阶段的动态处理和支撑阶段的地面推进有关。目前主动式矫形器的控制方式主要有基于自适应振荡器的控制、基于模型的控制、基于预定义步态模式动作以及基于强化学习与在线优化等四种方式\cite{yanReviewAssistiveStrategies2015}。R. Ronsse等人\cite{ronsseOscillatorbasedAssistanceCyclical2011c}开发的LOPES机器人最先使用了自适应非线性振荡器实现了髋、膝关节辅助。其通过使用自适应振荡器池来提取髋关节角度的相位和频率,然后将相位和髋关节角度输入到内核滤波器中以无延迟地估计髋关节角度。通过虚拟刚度计算出所需的关节扭矩以吸引髋关节达到预测的下一个位置。此外,与LOPES机器人类似,ALEX系列机器人\cite{winfreeDesignMinimallyConstraining2011,stegallVariableDampingForce2017,hidayahGaitAdaptationUsing2020}也基于髋关节的运动特征实现了在线步态分析并为使用者施加髋关节辅助扭矩。基于模型的控制方式主要赖于建立人机耦合物理模型来确定执行器的输出。在人形机器人领域,基于模型的控制系统策略通常需要充分考虑完整的机器人动力学。然而,在辅助机器人领域,模型有所不同,主要区别在于系统中引入了人类。其中,一个关键因素是对人与环境之间的接触进行建模\cite{youngStateArtFuture2017a}。 基于一个单自由度下肢辅助机器人,G. Aguirre-Ollinger等人\cite{aguirre-ollingerInertiaCompensationControl2012}提出了一种人体-外骨骼导纳模型,其生成的参考轨迹通过一个由LQR和积分项组成的闭环控制器进行跟踪。

\section{人-机器人交互界面研究现状}

\section{辅助机器人共享控制研究现状}

\section{研究现状总结及其存在的关键问题}

\section{研究内容及章节结构安排}

\subsection{二级节标题}

\subsubsection{三级节标题}

\paragraph{四级节标题}

\subparagraph{五级节标题}

本模板 \pkg{ustcthesis} 是中国科学技术大学本科生和研究生学位论文的 \LaTeX{}
模板, 按照《\href{https://gradschool.ustc.edu.cn/static/upload/article/picture/ce3b02e5f0274c90b9331ef50ae1ac26.pdf}
{中国科学技术大学研究生学位论文撰写手册}》(以下简称《撰写手册》)和
《\href{https://www.teach.ustc.edu.cn/?attachment_id=13867}
{中国科学技术大学本科毕业论文(设计)格式}》的要求编写。

Lorem ipsum dolor sit amet, consectetur adipiscing elit, sed do eiusmod tempor
incididunt ut labore et dolore magna aliqua.
Ut enim ad minim veniam, quis nostrud exercitation ullamco laboris nisi ut
aliquip ex ea commodo consequat.
Duis aute irure dolor in reprehenderit in voluptate velit esse cillum dolore eu
fugiat nulla pariatur.
Excepteur sint occaecat cupidatat non proident, sunt in culpa qui officia
deserunt mollit anim id est laborum.



\section{脚注}

Lorem ipsum dolor sit amet, consectetur adipiscing elit, sed do eiusmod tempor
incididunt ut labore et dolore magna aliqua.
\footnote{Ut enim ad minim veniam, quis nostrud exercitation ullamco laboris
  nisi ut aliquip ex ea commodo consequat.
  Duis aute irure dolor in reprehenderit in voluptate velit esse cillum dolore
  eu fugiat nulla pariatur.}
