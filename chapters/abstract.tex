% !TeX root = ../main.tex

\ustcsetup{
  keywords  = {康复辅助机器人,交互不确定性,人机交互,共享自主,概率建模},
  keywords* = {Rehabilitation Assistive Robot, Interaction Uncertainty, Human-Robot Interaction, Shared Autonomy, Probabilistic Modelling},
}

\begin{abstract}
人口老龄化已经成为我国近年来持续面临的主要挑战之一,康复辅助机器人有望为应对人口老龄化问题提供有力支持。在康复辅助机器人的人机交互系统设计中,需要综合考虑安全性和个体适应性是其区别于其他种类机器人的主要特征。然而,这两者往往是相悖的,过分强调安全排除一切可能造成风险的不确定性因素会导致机器人辅助决策倾向于保守进而降低使用者体验。另一方面,人类行为虽然在许多方面有明显的规则约束并且可预测,但它也始终是可变的、个体的,甚至是随机的,因此将更多使用者的自主行为引入交互系统中则会导致系统存在不可控的风险。通过有效应对人机交互中的不确定性,康复辅助机器人能够根据患者的需求和反馈进行个性化调整,从而实现康复辅助机器人在个体适应性和安全性之间的最佳平衡。目前,共享自主方法已被证实为应对上述问题的一种有效的解决方案。通过对人类行为进行概率建模,共享自主系统可以有效地将人机交互过程中的不确定性因素利用起来,进而自适应地调整系统的行为和响应,对于提高辅助系统性能和用户体验至关重要。

本文围绕共享自主方法,从机器人行为建模方法出发,对闭环人机交互过程中存在的不确定性因素进行了分析。通过搭建实验平台,深入探讨和研究了其在以下三个典型场景中的应用潜力与挑战:

(1)定制化人机交互接口在移动辅助机器人操控中的自适应交互指令解码。基于一个自主开发的新型柔性可穿戴人机交互接口,通过概率建模方法显式表达人机交互过程中的不确定性。本文分别开发了确定性直接数据映射解码方法,以及不确定性意图推理解码方法,用于交互指令生成。此外,我们将用户在使用确定性数据映射解码方式操控光标的先验表现集成到一个共享自主系统的非线性仲裁函数中,提出了一种自适应切换数据解码方法。实验证明,其不仅提高了使用者操控命令生成的准确性,同时保证了数据解码方法的动态性能以适应不同任务的要求。

(2)站立辅助机器人中被辅助对象的运动交互意图预测。针对人类在完成坐立运动时的移动速度不确定性,我们通过运动捕捉设备采集了真实场景下的人体坐立离线运动数据,建立了一个由概率化离散动态运动基元表征的先验技能库。根据对被辅助对象下肢运动的部分观测,本文将坐立运动时间估计看做一个系统参数辨识问题。基于期望最大化算法迭代优化计算模型时间缩放参数,实现了连续的人体坐立运动时间估计。所提出方法可以通过一个估计置信度水平量化指标嵌入一个共享自主系统中,进而实现辅助机器人的在线运动轨迹的稳定优化。

(3)偏瘫步态康复主动式膝关节辅助机器人自适应运动轨迹规划方法。围绕自主搭建的主动式膝关节矫形器原型样机,本文介绍了一种用于主动矫形器驱动器的在线对称步态轨迹规划方法。该方法基于节律型动态运动基元与一个耦合自适应非线性频率振荡器学习偏瘫患者健侧步态特征,实现了膝关节运动轨迹的在线的编码/解码以及步态相位的自适应延迟。此外,通过离线采集健康人群的步行示教数据,设计了一个带有先验步态技能库的共享自主系统。在非结构化环境下,该系统通过自适应地分析和仲裁来自健侧下肢的实时用户自主输入并用于规划位于患侧的主动矫形器的对称运动,在保证一定个体适应性的同时,减少了输入不确定性对矫形器辅助运动轨迹规划安全性的影响。
\end{abstract}

\begin{abstract*}
The aging of the population has become one of the major challenges our country has been continuously facing in recent years. Rehabilitation and assistive robots are expected to provide strong support in addressing the issue of population aging. In the design of human-machine interaction systems for rehabilitation and assistive robots, safety and individual adaptability must be comprehensively considered, which are the main features that distinguish from other robots. However, these two factors are often contradictory; overemphasizing safety to eliminate all potential risk factors can lead to a conservative approach that reduces the user experience. On the other hand, although human behavior is largely rule-bound and predictable, it is also variable, individual, and even random. Introducing more user autonomy into the interaction system can lead to uncontrollable risks. By effectively addressing uncertainties in human-machine interaction, rehabilitation assistive robots can make personalized adjustments based on the needs and feedback of patients, thus achieving an optimal balance between individual adaptability and safety. Currently, the shared autonomy approach has been proven to be an effective solution to the above issues. By probabilistic modeling of human behavior, shared autonomous systems can effectively use uncertain factors in human-machine interaction to adaptively adjust the system's behavior and response, which is crucial for enhancing the performance of assistive systems and user experience. 

This thesis focuses on the shared autonomy approach and, starting from the method of robot behavior modeling, analyzes the uncertainties in the closed-loop human-machine interaction process. By building experimental platforms, the potential applications and challenges in the following three typical scenarios are explored and studied in depth: 

(1) Customized human-machine interaction interfaces for adaptive interaction command decoding in the control of mobile assistive robots. Based on a self-developed new type of flexible wearable human-machine interaction interface, the uncertainty in the human-machine interaction process is explicitly represented using probabilistic modeling methods. This thesis has developed deterministic direct data mapping decoding methods and uncertainty-based intent inference decoding methods for generating interaction commands. Moreover, we integrate the prior performance of users controlling a cursor using deterministic data mapping decoding into the nonlinear arbitration function of a shared autonomous system, proposing an adaptive data decoding method for online adaptive switching. Experimental results prove that it not only improves the accuracy of command generation by users but also ensures the dynamic performance of the data decoding methods to meet the requirements of different tasks.

(2) In response to human mobility speed uncertainty when transitioning from sitting to standing, we have collected offline human sitting-standing motion data in real scenarios through motion capture devices and established a priori skill library characterized by probabilistic discretized dynamic motion primitives. By partially observing the assisted object's lower limb movements, this paper treats the estimation of sitting-standing motion time as a system parameter identification problem. The model's time scaling parameter is iteratively optimized based on the Expectation-Maximization algorithm, thus achieving continuous human sitting-standing motion time estimation. The proposed method can embed an estimate confidence level quantification index into a shared autonomous system, thereby stabilizing the optimization of the assisted robot's online motion trajectory.

(3) We introduces an online symmetric gait trajectory planning method for the actuator of an active knee orthosis prototype, which is developed for gait symmetry rehabilitation training. This method is based on periodic dynamic movement primitives and a coupled adaptive nonlinear frequency oscillator to learn the healthy side gait characteristics of hemiplegic patients, achieving online encoding/decoding of knee joint motion trajectories and the adaptive delay of gait phase. Additionally, by offline collecting walking demonstration data from healthy individuals, a shared autonomous system with a prior gait skill library was designed. In unstructured environments, the system adaptively analyzes and arbitrates real-time user inputs from the healthy side lower limb and uses them to plan the symmetrical movement of the active orthosis on the affected side. While ensuring a certain level of individual adaptability, it reduces the impact of input uncertainty on the safety of orthosis-assisted motion trajectory planning.
\end{abstract*}
