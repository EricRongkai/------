\chapter{总结与展望}
现阶段关于康复辅助机器的得智能化研究正在逐渐从``以机器为中心''向``以人为中心''过度。不仅仅需要在功能上满足使用者的需求,相较于传统的机器人研究工作,``以人为中心''的研究需要进一步从安全性、遵从性、舒适度、个体适应性以及灵活性等多方面的性能指标综合考虑。其中,在人机交互系统的设计研究中需要综合考虑安全性和个体适应性是康复辅助机器人区别于其他种类机器人的主要特征。然而这两者往往是相悖的,过分强调安全排除一切可能造成风险的不确定性因素会导致机器人倾向于保守进而降低使用者体验。另一方面,人类行为虽然在许多方面明显有规则约束并且可预测,但它也始终是可变的、个体的,甚至是随机的,将更多使用者的自主行为引入交互系统中则会导致系统存在不可控的风险。从机器人行为系统设计出发,第二章对闭环人机交互过程中由人类导致的交互不确定性来源进行了分析。我们认为,人机交互过程中的不确定性本身是一种信息,应该需要被正确的认识和对待,并且必须表示和处理所存在的不确定性,而不是盲目地过滤或忽略掉它。目前,康复辅助机器人中最为广泛的需求是对人类的物理运动的感知,而人类动作的不确定性是由感知、认知和执行等环节共同决定的,而采用了各类型传感器的人机交互接口是对人类的物理动作的量化表征。通过交互设备的感知数据实现更高层次的人类行为(意图)理解是处理人机交互问题的关键,在``以人为中心''的人机交互研究中,解决问题的目标应当不仅仅是一个简单明确的识别或回归算法,而是需要一个可以自适应处理交互中不确定的新框架。

共享自主方法已被证实为人机交互自适应问题的一种有效的解决方案。其通过实时感知与分析环境,使得机器人能够更加适应与人之间的物理交互,并具备持续进化的特性。在本文中,我们以共享自主方法为核心,深入探讨和研究了其在以下三个典型场景中的应用潜力与挑战:定制化人机交互接口在移动辅助机器人操控中的自适应交互指令解码、站立辅助机器人中被辅助对象的运动交互速度自适应感知、以及用于偏态步态康复的主动式膝关节辅助机器人自适应运动轨迹规划方法。其中在第一项研究中,我们针对所设计的新型可穿戴人机交互接口提出了一种自适应切换数据解码方法,它将用户的使用确定性数据映射模式操控光标的先验表现信息集成到一个共享自主系统的非线性仲裁函数中,实现两种数据解码方法的自适应切换。其不仅提高了使用者操控命令生成的准确性,同时保证了数据解码方法的动态性能以适应不同任务的要求。在第二项研究中,针对人类在完成坐立运动时的速度不确定性,我们通过离线采集真实场景下的人体坐立离线运动数据,建立了一个由概率化离散动态运动基元先验坐立运动技能库。通过将在部分观测下的运动时间估计看做一个系统参数辨识问题,我们基于期望最大化算法迭代优化坐立动作模型的时间缩放系数,实现了连续的运动时间估计。此外,所设计站立动作速度估计方法通过一个估计置信度水平量化指标可以嵌入一个共享自主系统中,实现辅助机器人的在线运动轨迹的稳定优化或变刚度控制。在第三项研究中,围绕一个用于偏瘫患者步态对称性康复训练的主动式膝关节矫形器原型样机,我们详细地介绍了一种通过学习偏瘫健侧步态特征的在线对称步态轨迹生成方法。该方法通过融合节律型动态运动基元与一个自适应非线性频率振荡器,实现了在线的步态运动周期轨迹的编码与解码,同时实现了下肢健侧和患侧的步态相位自适应延迟。此外,通过对健康人群的离线步行任务采集数据,设计了一个带有先验步态技能库的共享自主模块,以自适应地分析和仲裁来自健侧的实时用户输入,从而最大程度地减轻输入不确定性的影响。

综合来看,``以人为中心''的康复辅助机器人交互研究涉及到多个领域的知识,目前相关理论研究尚不成熟并且仍处于探索阶段。针对不同的应用场景以及交互感知方式,实现对人类动作(意图)的理解并将其融入一个共享自主系统中仍然是特异的。在未来,随着康复辅助机器人的智能化研究从感知智能逐步迈向认知智能阶段,如何充分利用更丰富的感知信息以及更强大的计算资源设计共享自主系统,是实现通用辅助机器人在实际生活中落地应用的重要研究方向。