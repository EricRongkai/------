% !TeX root = ../main.tex

\begin{acknowledgements}

行文至此,二十余载的求学生涯也似乎是告一段落。这一路走来,如今已而立之年,遇到过太多的人想要去感谢。曾经无数次想过要如何写下这一部分内容,然而思绪万千,却又迟迟难以落笔。

首先,我要向我尊敬的导师宋全军研究员表达深深的感激之情。五年前我们第一次见面,我至今记忆犹新。感谢您曾给予我这样一个并非那么优秀的学生攻读博士学位的机会,以及对我的足够信任和包容,使我有机会自由探索我所感兴趣的研究课题。还记得您曾在我入学时强调,机器人研究的目标不仅在于发表科研论文,更重要的是要落地实践,解决现实需求。这一点我将永远铭记在心,并在今后的研究道路中为之努力践行。特别感谢实验室的葛运建研究员、刘明教授、宋博研究员、聂余满研究员对我在学术研究和论文写作方面的悉心指导。您们无私的帮助和付出使我从一个不成熟的研究生逐步成长为一个对机器人领域有所理解的学者。感谢潘宏青主任和马婷婷老师对我所开展的研究工作给予的大力支持,无比怀念每一个一起交流科研方向,开启头脑风暴的日子。自从入学以来,您们不仅带领我参与项目申请,也带我参加了诸多学术交流活动,这些经历使我对科研工作有了更为深刻地认识和理解。感谢李皓老师、曹平国老师、徐湛楠老师对于我搭建实验平台的鼎力相助。在我遇到困难或问题时,您们总是乐于提供解决方案并帮助我克服困难,让我在这个过程中学习到了宝贵的知识和经验。感谢实验室的大管家孙玉苹老师,您对学生的帮助和关心让我们在实验室就像在家中一样温暖。感谢班主任夏文彬老师,您总是倾尽全力协助我们解决问题,帮我们度过了一个又一个关卡。感谢我的硕士导师中国矿业大学(北京)佟丽娜老师、中科院自动化研究所彭亮老师,感谢您们带领我走入了人-机器人交互的研究领域。此外,还要感谢腾讯Robotics X实验室的陈立鹏,李景辰学长,给了我参与人形机器人项目研究的交流机会。

我衷心感谢我的父亲刘有贵先生,母亲王晓丽女士,以及始终支持我完成学业的每一位亲人。你们为我营造了一个温馨的家庭氛围,并且为我的成长与教育倾注了无数心血与牺牲。抱歉我的任性,还从未真正担负起作为子女和晚辈应尽的责任。感谢我的妻子,罗诗颖女士。从二十岁的相识、相恋再到步入婚姻,十年的时间对你的陪伴和支持亏欠了太多,愿用余生回报你这些年来的理解和默默付出。

感谢在智能感知技术研究中心遇到的每一位同学,能与众多杰出的同伴共事,我感到万分荣幸。特别感谢在112实验室与我一起并肩战斗的赵欣彦同学和陈勇同学。愿你们在未来攻读博士的科研之路上,一切顺利,硕果累累。

最后,感谢一路坚持的自己。此间翻山而过,不求今后皆是明月清风与坦荡通途,只愿无论磨难,一往无前,初心不改。
\newline
\newline
\begin{flushright}
    作者:刘镕恺\\

    2024年5月18日\\

    安徽·合肥·科学岛
\end{flushright}

\end{acknowledgements}
