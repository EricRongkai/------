% !TeX root = ../main.tex

\chapter{绪论}

\section{研究背景与意义}
人口老龄化问题成为我国近年来持续面临的挑战,据《2021年度国家老龄事业发展公报》\cite{2021NianDuGuoJiaLaoLingShiYeFaZhanGongBao}的数据指出,截至2021年末,全国60周岁及以上老年人口26736万人,占总人口的18.9\%;全国65周岁及以上老年人口20056万人,占总人口的14.2\%。全国65周岁及以上老年人口抚养比20.8\%。其中《第四次中国城乡老年人生活状况抽样调查成果》中的数据显示,2020 年中国失能老年人达到 4200 万,空巢和独居老年人已达到 1.18 亿。由于祖辈和子代两地分居,子代对祖辈的照顾多来自于经济支持,而缺少对于生活照护、情感陪伴等家庭养老支持,其严重弱化了独居与失能、半失能老年人的生活状况。面向助老助残领域的康复辅助机器人研究对于减轻社会养老负担,提升失能老年人生活独立性和自信心具有重要意义。

康复辅助机器人一般用于辅助或扩展人类的运动和/或认知能力,主要面向对象为:体弱的老人、截肢者和其他患有脊髓损伤、中风等疾病的人群。 近年来,我国先后颁布了《中国制造2025》、《“健康中国2030”规划纲要》等文件,重点支持康复辅助机器人产业发展。辅助机器人一般需要其拥有智能化和鲁棒性以维持系统的安全高效性和灵活性,通过集成在线信息处理机制,机电一体化和先进的人机交互接口与人进行物理或者其他感官接触。尽管近年来机器人技术以及模式识别技术取得了巨大的进展,在辅助机器人领域,我们仍远未为机器人提供完全的自主权,从而使它们能够在动态变化的环境中能够处理更多的情况。

近年来,以机器学习和深度学习为代表的机器智能技术的快速发展取得显著成果。然而,随着数据量增加,``智能提升''逐渐减弱。与此同时,人工智能系统与人类之间的互动关系变得更加广泛和复杂。相较于人类智能,机器智能通常根据当前对外部环境的感知给出确定性的决策,灵活程度低且容易受到感知系统输入不确定性的影响导致决策失败。然而,机器决策在动态变化的环境下往往可以产生高维度的操控指令并作出最优的决策,进而保证系统的整体安全高效。在康复辅助机器人系统中,使用者一般都存在显著认知和运动障碍,通常仅能使用有限离散的交互接口导致基于交互界面的人类决策输入存在更高程度的交互不确定性,容易造成意外情况出现。然而,人类在大部分情况下的决策往往灵活度高更能代表使用者内部期望,这提出了一个核心问题:在需要保证安全可靠性的人-辅助机器人交互系统中,应该怎样处理用户的交互输入不确定性?





\section{康复辅助机器人研究现状}

\section{人-机器人交互方法研究现状}

\section{辅助机器人共享控制研究现状}

\section{研究现状总结及其存在的关键问题}

\section{研究内容及章节结构安排}

\subsection{二级节标题}

\subsubsection{三级节标题}

\paragraph{四级节标题}

\subparagraph{五级节标题}

本模板 \pkg{ustcthesis} 是中国科学技术大学本科生和研究生学位论文的 \LaTeX{}
模板, 按照《\href{https://gradschool.ustc.edu.cn/static/upload/article/picture/ce3b02e5f0274c90b9331ef50ae1ac26.pdf}
{中国科学技术大学研究生学位论文撰写手册}》(以下简称《撰写手册》)和
《\href{https://www.teach.ustc.edu.cn/?attachment_id=13867}
{中国科学技术大学本科毕业论文(设计)格式}》的要求编写。

Lorem ipsum dolor sit amet, consectetur adipiscing elit, sed do eiusmod tempor
incididunt ut labore et dolore magna aliqua.
Ut enim ad minim veniam, quis nostrud exercitation ullamco laboris nisi ut
aliquip ex ea commodo consequat.
Duis aute irure dolor in reprehenderit in voluptate velit esse cillum dolore eu
fugiat nulla pariatur.
Excepteur sint occaecat cupidatat non proident, sunt in culpa qui officia
deserunt mollit anim id est laborum.



\section{脚注}

Lorem ipsum dolor sit amet, consectetur adipiscing elit, sed do eiusmod tempor
incididunt ut labore et dolore magna aliqua.
\footnote{Ut enim ad minim veniam, quis nostrud exercitation ullamco laboris
  nisi ut aliquip ex ea commodo consequat.
  Duis aute irure dolor in reprehenderit in voluptate velit esse cillum dolore
  eu fugiat nulla pariatur.}
