% !TeX root = ../main.tex

\begin{publications}

\section*{已发表和待录用论文}

\begin{enumerate}
\item \textbf{R. Liu}, T. Ma, N. Yao, H. Li, X. Zhao, Y. Wang, H. Pan, Q. Song., “Adaptive Symmetry Reference Trajectory Generation in Shared Autonomy for Active Knee Orthosis,” \textit{IEEE Robotics and Automation Letters (RAL)}, vol. 8, no. 6, pp. 3118–3125, Jun. 2023. (SCI, IF:5.2, TOP, 对应本文第五章内容)
\item \textbf{R. Liu}, Q. Song, T. Ma, H. Pan, H. Li and X. Zhao, ``Mapping Movement of Shoulder to Commands: A Non-invasive Wearable Body-Machine Interface Based on Soft Sensors'' \textit{Journal of Neural Engineering (JNE)}. Accept (SCI, IF:4.0, 对应本文第三章内容)
\item X. Zhao, \textbf{R. Liu}, T. Ma, H. Li, and Q. Song, “Real-time Gait Phase Estimation Based on Multi-source Flexible Sensors Fusion,” in Proceedings of the 2023 3rd International Conference on Robotics and Control Engineering, Nanjing China: ACM, May 2023, pp. 113–118. (EI, Best Paper, 指导硕士研究生发表,对应本文第五章内容)
\item Y. Wang, Q. Song, T. Ma, Y. Chen, H. Li, and \textbf{R. Liu}, “Transformation classification of human squat/sit-to-stand based on multichannel information fusion,” International Journal of Advanced Robotic Systems, vol. 19, no. 4, p. 172988062211037, Jul. 2022. (SCI, IF:2.3, 对应本文第四章内容)
\end{enumerate}

\section*{学术会议报告}
\begin{enumerate}
\item ``Adaptive Symmetry Reference Trajectory Generation in Shared Autonomy for Active Knee Orthosis'' – 2023 IEEE/RSJ International Conference on Intelligent Robots (IROS 2023), Detroit, Michigan, USA. (Oral+Poster)
\end{enumerate}

\section*{准备中的论文}
\begin{enumerate}
\item 与腾讯公司机器人实验室合作论文, ``An Online Optimization Method for Assisted Trajectories of Humanoid Robots Based on Sit-to-Stand Motion Prediction,'' 准备中,拟投稿\textit{IEEE Transactions on Medical Robotics and Bionics}. (对应本文第四章内容)
\end{enumerate}

\section*{发明专利}
\begin{enumerate}
\item 宋全军,\textbf{刘镕恺},马婷婷,潘宏青,“非侵入式体机交互接口及用户意图推理方法”(公开号:CN 118070903 A,实质审查);
\item 宋全军,\textbf{刘镕恺},马婷婷,潘宏青,李皓,赵欣彦,“一种步态信息处理方法和矫形器”(公开号:CN 118114070 A,实质审查);
\item 宋全军,赵欣彦,\textbf{刘镕恺},马婷婷,潘宏青,李皓,“一种用于偏瘫康复膝关节矫形器系统和使用方法”(申请号:202410439532.9,实质审查);
\end{enumerate}

% \section*{参与的科研项目}
% \begin{enumerate}
% \item 基于大数据的自然交互意图理解和智能输入-国家重点研发计划 (参与)
% \item 脑卒中康复机器人-国家重点研发计划 (参与)
% \item 面向机器人交互的柔性应变传感器研制与应用-安徽省重点研发计划 (参与)
% \item 智能轮椅助行车技术研发与产业化-中国科学技术大学“双创基金” -5万元(负责人)
% \end{enumerate}

\end{publications}
