% !TeX root = ../main.tex

\ustcsetup{
  keywords  = {康复辅助机器人,人-机器人交互,交互不确定性,意图推理,共享自主},
  keywords* = {Rehabilitation Assistive Robot, Human-Robot Interaction, Interaction Uncertainty, Intent Inference, Shared Autonomy},
}

\begin{abstract}
人口老龄化已经成为我国近年来持续面临的主要挑战之一,康复辅助机器人有望为应对人口老龄化问题提供有力支持。在康复辅助机器人的人机交互系统设计中,需要综合考虑安全性和个体适应性是其区别于其他种类机器人的主要特征。然而,这两者往往是相悖的,过分强调安全排除一切可能造成风险的不确定性因素会导致机器人辅助决策倾向于保守进而降低使用者体验。另一方面,人类行为虽然在许多方面有明显的规则约束并且可预测,但它也始终是可变的、个体的,甚至是随机的,因此将更多使用者的自主行为引入交互系统中则会导致系统存在不可控的风险。通过有效应对人机交互中的不确定性,康复辅助机器人能够根据患者的需求和反馈进行个性化调整,从而实现康复辅助机器人在个体适应性和安全性之间的最佳平衡。目前,共享自主方法已被证实为应对上述问题的一种有效的解决方案。通过对人类行为进行分析和概率建模,共享自主系统可以有效地将人机交互过程中的不确定性因素利用起来,进而自适应地调整系统的行为和响应,对于提高辅助系统性能和用户体验至关重要。

基于运动跟踪方式的辅助机器人交互感知系统是目前应用最广泛和成熟的方案。相比于其他的感知方法,在交互中对使用者产生的物理动作进行捕捉需要解决的一个主要问题是:如何在感知数据中区分哪些信息是由使用者期望产生(模式),哪些信息是非期望产生的(噪声)。本文从机器人行为系统设计方法出发,对闭环人机交互过程中存在的不确定性来源进行了分析。并针对人机交互过程中的人类动作不确定性,通过搭建实验平台,深入探讨和研究了考虑交互不确定性的共享自主方法,在康复辅助机器人三个不同康复阶段的典型应用中的潜力与挑战:

(1)定制化人机交互接口在移动辅助机器人操控中的自适应交互指令解码。研究针对使用者肩部交互动作,基于一个自主开发的柔性可穿戴体-机交互接口,通过概率建模方法显式表达使用肩部进行人机交互的不确定性,分别开发了确定性直接数据映射解码方法,以及基于意图推理的解码方法用于交互指令生成。此外,通过将用户在使用确定性数据映射解码方式操控光标的先验表现集成到一个共享自主系统的非线性仲裁函数中,研究提出了一种自适应切换数据解码方法。通过一组光标控制和虚拟轮椅操控实验表明,该方法不仅提高了体-机交互接口中使用者操控命令生成的准确性,同时保证了数据解码方法的动态性能以适应不同任务的要求。

(2)坐立辅助机器人中被辅助对象的运动交互意图预测。研究针对被辅助对象下肢坐立运动交互动作,围绕其在完成坐立运动时的移动速度不确定性,通过采集人体坐立关节运动数据,建立了一个由概率化离散动态运动基元表征的人体坐立运动关节轨迹先验模型。根据对被辅助对象下肢关节运动的部分观测数据,研究将坐立运动时间估计看做一个系统参数辨识问题,基于期望最大化算法迭代优化计算模型时间缩放参数,实现了连续的人体坐立运动时间估计。通过理论分析与实验证明,所设计的站立动作速度估计方法可以仅依靠三组不同速度下的示教轨迹实现在线意图推理,极大地提高了数据利用率。此外其通过一个估计置信度水平量化指标嵌入一个共享自主系统中,进而实现辅助机器人运动轨迹的在线稳定优化或变刚度控制。

(3)偏瘫步态康复主动式膝关节辅助机器人自适应运动轨迹规划方法。针对偏瘫康复训练步行交互动作,围绕偏瘫下肢健侧步态运动输入不确定性,基于自主搭建的主动式膝关节矫形器原型样机,研究介绍了一种用于主动矫形器驱动器的在线对称步态轨迹学习/规划方法。该方法基于节律型动态运动基元与一个耦合自适应非线性频率振荡器学习偏瘫患者健侧步态特征,实现了膝关节运动轨迹的在线的编码/解码以及步态相位的自适应延迟。此外,通过离线采集健康人群的步行示教数据,设计了一个带有先验步态技能库的共享自主系统。在非结构化环境下,该系统通过自适应地分析和仲裁来自健侧下肢的实时用户自主输入,用于规划位于患侧的主动矫形器的对称运动。在保证一定个体适应性的同时,减少了下肢健侧运动轨迹输入不确定性对矫形器运动轨迹规划安全性的影响。

\end{abstract}

\begin{abstract*}
The aging of the population has become one of the major challenges our country has been continuously facing in recent years. Rehabilitation and assistive robots are expected to provide strong support in addressing the issue of population aging. In the design of human-machine interaction systems for rehabilitation and assistive robots, safety and individual adaptability must be comprehensively considered, which are the main features that distinguish from other robots. However, these two factors are often contradictory; overemphasizing safety to eliminate all potential risk factors can lead to a conservative approach that reduces the user experience. On the other hand, although human behavior is largely rule-bound and predictable, it is also variable, individual, and even random. Introducing more user autonomy into the interaction system can lead to uncontrollable risks. By effectively addressing uncertainties in human-machine interaction, rehabilitation assistive robots can make personalized adjustments based on the needs and feedback of patients, thus achieving an optimal balance between individual adaptability and safety. Currently, the shared autonomy approach has been proven to be an effective solution to the above issues. By probabilistic modeling of human behavior, shared autonomous systems can effectively use uncertain factors in human-machine interaction to adaptively adjust the system's behavior and response, which is crucial for enhancing the performance of assistive systems and user experience. 

Assistive robotic interaction perception systems based on motion tracking are currently the most widely used and mature solutions. Compared to other perception methods, one major issue that needs to be addressed in capturing the physical actions of the user during interaction is how to distinguish in the perception data which information is desired by the user (pattern) and which is undesired (noise). This paper starts from the design method of robotic behavior systems and analyzes the sources of uncertainty in the closed-loop human-machine interaction process. Furthermore, considering the uncertainty of human actions in the human-machine interaction process, by building an experimental platform, this study deeply investigates and explores the shared autonomy method that considers interactive uncertainty, and the potential and challenges in its typical applications in three different rehabilitation phases of rehabilitative assistive robots.

(1) Customized human-machine interaction interfaces for adaptive interaction command decoding in the control of mobile assistive robots. The research focuses on the user's shoulder interaction actions and is based on a self-developed flexible wearable body-machine interaction interface. Through the probabilistic modeling method, the uncertainty in the human-machine interaction process is explicitly expressed. A deterministic direct data mapping decoding method and a decoding method based on uncertainty intent inference have been developed separately for the generation of interaction instructions. Moreover, we integrate the prior performance of users controlling a cursor using deterministic data mapping decoding into the nonlinear arbitration function of a shared autonomous system, proposing an adaptive data decoding method for online adaptive switching. Experimental results prove that it not only improves the accuracy of command generation by users but also ensures the dynamic performance of the data decoding methods to meet the requirements of different tasks.

(2) In response to human mobility speed uncertainty when transitioning from sitting to standing, we have collected offline human sitting-standing motion data in real scenarios through motion capture devices and established a priori skill library characterized by probabilistic discretized dynamic motion primitives. By partially observing the assisted object's lower limb movements, this paper treats the estimation of sitting-standing motion time as a system parameter identification problem. The model's time scaling parameter is iteratively optimized based on the Expectation-Maximization algorithm, thus achieving continuous human sitting-standing motion time estimation. The proposed method can embed an estimate confidence level quantification index into a shared autonomous system, thereby stabilizing the optimization of the assisted robot's online motion trajectory.

(3) We introduces an online symmetric gait trajectory planning method for the actuator of an active knee orthosis prototype, which is developed for gait symmetry rehabilitation training. This method is based on periodic dynamic movement primitives and a coupled adaptive nonlinear frequency oscillator to learn the healthy side gait characteristics of hemiplegic patients, achieving online encoding/decoding of knee joint motion trajectories and the adaptive delay of gait phase. Additionally, by offline collecting walking demonstration data from healthy individuals, a shared autonomous system with a prior gait skill library was designed. In unstructured environments, the system adaptively analyzes and arbitrates real-time user inputs from the healthy side lower limb and uses them to plan the symmetrical movement of the active orthosis on the affected side. While ensuring a certain level of individual adaptability, it reduces the impact of input uncertainty on the safety of orthosis-assisted motion trajectory planning.
\end{abstract*}
