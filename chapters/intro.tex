% !TeX root = ../main.tex

\chapter{绪论}

\section{研究背景与意义}
人口老龄化问题成为我国近年来持续面临的挑战,据《2021年度国家老龄事业发展公报》\cite{2021NianDuGuoJiaLaoLingShiYeFaZhanGongBao}的数据指出,截至2021年末,全国60周岁及以上老年人口26736万人,占总人口的18.9\%;全国65周岁及以上老年人口20056万人,占总人口的14.2\%。全国65周岁及以上老年人口抚养比20.8\%。其中《中国养老服务蓝皮书(2012—2021)》\cite{YangZhongGuoYangLaoFuWuLanPiShu201220212022}中的数据显示,2015年我国失能、部分失能老年人达到4063万人,占老年人口的18.3\%。与此同时,到2025年我国失能总人口预计将上升到7279.22万人,2030年将达1亿人,空巢和独居老年人已达到 1.18 亿。由于祖辈和子代两地分居,子代对祖辈的照顾多来自于经济支持,而缺少对于生活照护、情感陪伴等家庭养老支持,其严重弱化了独居与失能、半失能老年人的生活状况。面向助老助残领域的康复辅助机器人研究对于减轻社会养老负担,提升失能老年人生活独立性和自信心具有重要意义。

康复辅助机器人一般用于辅助或扩展人类的运动和/或认知能力,主要面向对象为:体弱的老人、截肢者和其他患有脊髓损伤、中风等疾病的人群。 近年来,我国先后颁布了《中国制造2025》\cite{ZhongGuoZhiZao2025}、《“健康中国2030”规划纲要》\cite{LiuYangJianKangZhongGuo2030GuiHuaGangYao}等文件,重点支持康复辅助机器人产业发展。辅助机器人一般需要其拥有智能化和鲁棒性以维持系统的安全高效性和灵活性,通过集成在线信息处理机制,机电一体化和先进的人机交互接口与人进行物理或者其他感官接触。尽管近年来机器人技术以及模式识别技术取得了巨大的进展,在辅助机器人领域,我们仍远未为机器人提供完全的自主权,从而使它们能够在动态变化的环境中能够处理更多的情况。

近年来,以机器学习和深度学习为代表的机器智能技术的快速发展并取得显著成果。然而,随着数据量增加,``智能提升''逐渐减弱。与此同时,人工智能系统与人类之间的互动关系变得更加广泛和复杂\cite{ZhaoQianTanKongZhiZhongDeGongXiangXinXiHeGongXiangZiZhu2021}。相较于人类智能,机器智能通常根据当前对外部环境的感知给出确定性的决策,灵活程度低且容易受到感知系统输入不确定性的影响导致决策失败。然而,机器决策在动态变化的环境下往往可以产生高维度的操控指令并作出最优的决策,进而保证系统的整体安全高效。然而,在大多数辅助机器人应用中,通常都由人类操作者操作或监督机器人。在康复辅助机器人系统中,使用者一般都存在显著认知和运动障碍,通常仅能使用有限离散的交互接口导致基于交互界面的人类决策输入存在更高程度的交互不确定性,容易造成意外情况出现。相较于机器智能,人类操作者能够提供优越的情境感知、逻辑和解决问题的能力,但这也提出了一个核心问题:在需要保证安全可靠性的人-辅助机器人交互系统中,我们应当怎样处理用户的交互输入的不确定性。

在物理人-机器人交互领域,机器自主是一种手段,而不是目标,自主级别因应用领域的不同存在差异。与智能控制器共享机器人系统的控制,可以让人类在执行任务时减少认知和身体上的工作量,在另一方面机器智能又能够在一定程度上纠正使用者的错误输入。在``通用人工智能''时代真正到来之前,人机混合将长期存在于辅助机器人系统中\cite{ZhangMianXiangRenJiXuGuanJueCeDeHunHeZhiNengFangFaYanJiu2021}。因此,如何在康复辅助机器人中动态处理人类交互输入不确定性,对于提高康复辅助机器人系统的鲁棒性和安全性,减轻使用者负担,促进相关技术应用落地具有重要意义。

\section{康复辅助机器人研究现状}
康复辅助机器人技术旨在为残疾患者提供支持,使他们能够更独立地进行日常生活活动(ADL)。例如,移动、抓握和处理物体、进食等。目前该领域的的研究涉及机械,信息,控制,计算机,医学等多个学科,是当前机器人研究领域的热点。目前的辅助设备如电动轮椅、辅助机械臂、上肢或下肢外骨骼机器人和主动矫形器,对于帮助那些有严重运动障碍的人改善独立生活能力、减轻家庭照护负担至关重要。
\subsection{主动矫形器与外骨骼式辅助机器人}
主动矫形器与外骨骼机器人经过数代发展,涌现出多种拥有不同驱动方式和控制方法的系统。其中物理人机交互控制系统是其核心部分,需要系统能够实时感知用户的运动意图和姿态,并根据这些信息控制机器人的动作,常用的控制方法包括基于模型的控制、基于传感器的控制和混合控制等。目前,相关研究更加强调在多学科领域的整合,包括生物力学、机器人工程、神经科学等多个学科的交叉,推动了该领域的不断创新。相较于外骨骼机器人,主动式矫形器和智能假肢主要用于帮助患者恢复行走能力,改善步态,减轻关节负担等,通常采用被动式或半主动式驱动方式,即根据患者的步态自动调整其支撑力和运动模式。而下肢外骨骼机器人则采用全主动式驱动方式,可以根据患者的需求和外部指令进行精确的运动控制,并用于康复训练、提高运动能力、增强体力等。

相对于需要携带自己的电池和执行器的完整外骨骼来说,主动矫形器作为一种靶向型外骨骼机器人普遍都有更低质量和转动惯量,因此驱动器的输出功率可以更多用于为用户提供辅助而不是补偿自身重量\cite{collinsReducingEnergyCost2015,zhangHumanintheloopOptimizationExoskeleton2017a}。其一般多为单关节结构,在保留使用者一部分自由活动能力的前提下通过驱动器辅助特定关节来完成运动动作,其可分为髋关节、膝关节或踝关节辅助系统\cite{malcolmExperimentalStudyRole2009}。在步行中,膝关节主要是一个自由阻尼关节,在摆动阶段几乎处于锁定状态,而在支撑阶段则与之相反;髋关节和踝关节主要与摆动阶段的动态处理和支撑阶段的地面推进有关。目前主动式矫形器的控制方式主要有基于自适应振荡器的控制、基于模型的控制、基于预定义步态模式动作以及基于强化学习与在线优化等四种方式\cite{yanReviewAssistiveStrategies2015}。

图\ref{fig:1-1}展示了部分高校和公司研发的外骨骼式辅助机器人。Ronsse等人\cite{ronsseOscillatorbasedAssistanceCyclical2011c}开发的LOPES机器人最先使用了自适应非线性振荡器实现了髋、膝关节辅助。其通过使用自适应振荡器池来提取髋关节角度的相位和频率,然后将相位和髋关节角度输入到内核滤波器中以无延迟地估计髋关节角度。通过虚拟刚度计算出所需的关节扭矩以吸引髋关节达到预测的下一个位置。此外,与LOPES机器人类似,ALEX系列机器人\cite{winfreeDesignMinimallyConstraining2011,stegallVariableDampingForce2017,hidayahGaitAdaptationUsing2020}也基于髋关节的运动特征实现了在线步态分析并为使用者施加髋关节辅助扭矩。基于模型的控制方式主要赖于建立人机耦合物理模型来确定执行器的输出。在人形机器人领域,基于模型的控制系统策略通常需要充分考虑完整的机器人动力学。然而,在辅助机器人领域,模型有所不同,主要区别在于系统中引入了人类。其中,一个关键因素是对人与环境之间的接触进行建模\cite{youngStateArtFuture2017a}。 基于一个单自由度下肢辅助机器人,Aguirre-Ollinger等人\cite{aguirre-ollingerInertiaCompensationControl2012}提出了一种人体-外骨骼导纳控制模型,其生成的参考轨迹通过一个由LQR和积分项组成的闭环控制器进行跟踪。通过将外骨骼机器人的动力学模型分解为摆动相和支撑相,王立坤等人\cite{WangXiaZhiZhuLiWaiGuGeJiQiRenKongZhiXiTongRenJiGongRongCeLueYanJiu2019}提出一种基于混杂自动机的下肢助力外骨骼机器人动力学系统。为了实现一个基于串联弹性单元驱动的外骨骼动力学补偿和用户特定的辅助,Vantilt等人\cite{vantiltModelbasedControlExoskeletons2019}通过对完整的外骨骼动力学和与环境接触的建模,实现了用户由坐到站的辅助。

\begin{figure}[h]
  \centering
  \includegraphics[width=1\textwidth]{1-fig-1.pdf}
  \caption{部分高校和公司研发的主动矫形器与外骨骼式辅助机器人}
  \label{fig:1-1}
  % \note{Exoskeleton robots developed by some universities and companies}
\end{figure}

建立精确的人机耦合动力学模型往往较为困难,预定义轨迹由于实现较为简单,是目前商业应用中最常见的模式。基于预定义轨迹的控制器通常用于步态训练器和完全截瘫患者的外骨骼,并通过位置或阻抗控制来实现轨迹跟随。预定义参考轨迹通常来自于健康人群的运动数据或者直接通过人工设置,但是其通常无法实现使用者的个性化运动模式。例如Zhong等人\cite{zhongGaitSymmetryEnhancement2022}基于预定义力矩辅助曲线和一个基于串联弹性单元驱动器的膝-踝-足矫形器实现了下肢步态辅助。近年来,由于系统允许佩戴者自由地在无约束环境下进行移动,围绕单关节靶向式外骨骼设备,基于在线学习与优化方式的控制器逐渐成为主流研究方向。Kawamoto等人\cite{kawamotoModificationHemiplegicCompensatory2015,kawamotoDevelopmentAssistController2014a}设计了一个单腿版本的混合辅助肢体(HAL),在摆动阶段为受影响的肢体提供帮助,其中使用一个运动缓存器来在线存储未受影响肢体的运动数据用规划机器人运动轨迹。Huang和Peng等人\cite{huangLearningbasedWalkingAssistance2018,pengDataDrivenReinforcementLearning2020}提出了一种基于强化学习的步行辅助控制策略。他们通过将未受影响的一方作为领导者,将外骨骼作为追随者来模拟机器人控制系统。相较于简单的预定义轨迹,人在环中优化算法可以实时为参与者个性化辅助机器人控制参数,避免了需要对人机耦合系统进行精确建模这一技术难点,为辅助机器人个性化提供了可能。卡内基梅隆大学和哈佛大学\cite{dingHumanintheloopOptimizationHip2018,zhangHumanintheloopOptimizationExoskeleton2017a,awadSoftRoboticExosuit2017}所开发的一系列绳驱动柔性外骨骼系统已证明使用辅助髋部和踝关节运动的外部设备可减少使用者在步行时的代谢消耗。

\subsection{移动辅助机器人}
相对于可穿戴式辅助机器人用于移动支持,轮椅是最常用的辅助设备\cite{worldhealthorganizationGuidelinesProvisionManual2008b}。根据国家卫健委和中国残疾人联合公布的数据,我国失
能、半失能老年人口高达4200万人,肢体残疾人人数约为2472万人,预计轮椅的实际需求用户接近4000万人。那些由于认知、运动或感觉障碍而受影响的个体,无论是由于残疾还是疾病,通常依赖于电动轮椅来完成移动任务。为了满足那些在操纵传统轮椅方面面临困难或无法操纵的个体的需求,一些研究人员已经引入最初用于移动机器人的技术,从而研发出智能轮椅。此外,由于一些残疾人无法使用传统的操纵杆来导航,因此需要采用替代的控制系统,如头部操纵杆、下巴操纵杆、抽吸器以及脑控技术\cite{kimLiteratureReviewSmart2023}。这类移动辅助设备通常具有简单而易于操作的结构,不仅在为失能用户提供身体活动方面发挥着关键作用,同时也显著提升了他们的社会参与能力。

目前,国际上研发智能轮椅的主要企业有 Permobil、Lucy Mobility 等,国内也有相关企业开始布局该领域,如椅夫健康、邦邦机器人,图\ref{fig:1-2}展示了部分国内外商业智能移动辅助机器人。这些企业在技术研发、产品创新和市场推广方面具有较强实力,但是需要进一步挖掘用户需求并结合先进机器人技术提高轮椅智能化水平以应对市场竞争。轮椅行业技术升级的趋势已经显现,智能化人性化将成为未来市场的主流。相关政策也支持进一步提高老年人残疾人用品适用性,并提出要围绕特殊人群消费品发展需求,加大人体工效基础研究、技术研发和标准制定力度。

智能轮椅本质上为一个带有一系列如激光雷达、摄像头、红外线传感器等传感器和计算机控制系统的移动差速机器人,其智能体现在感知和操控两个方面\cite{kimLiteratureReviewSmart2023}。首先,在感知方面,是以自动避障,自动导航以及地形自适应等功能为代表的安全导航功能。近些年来,自动驾驶汽车领域学术和工业研究的爆炸式增长,使得自动驾驶技术在各个方面取得了显著进步,并被广泛应用于提升辅助轮椅的自动化水平。这一方面比较代表性的研究有MIT-CSAIL实验室研发的自主移动轮椅项目\cite{walterFrameworkLearningSemantic2014},其通过环境感知和无线室内定位系统开发了一款可以使用语音命令操控的自动导航轮椅,从而增强普通电动轮椅的功能。美国密歇根大学安娜堡Vulcan等人\cite{fosterReflectanceFieldMap2023,parkDiscretetimeDynamicModeling2017}开发的智能轮椅采用了一种混合空间语义层次结构(HSSH)算法用于进行导航空间知识表示,其能够实现高效学习和自然的人机交互。新加坡国立大学开发的全自动驾驶助行车\cite{SelfdrivingScooterUnveiled}可以用于在大型车辆难以通行的较小和较窄的道路上行驶。iBOT智能轮椅设计了四轮驱动底盘\cite{MobiusMobilityNext},利用地形跟踪技术实现了陡坡和各种地形的自适应,其座椅角度可根据坡度自动调节进而令使用者保持稳定。
\begin{figure}[h]
  \centering
  \includegraphics[width=1\textwidth]{1-fig-2.pdf}
  \caption{国内外商业公司研发智能移动辅助机器人}
  \label{fig:1-2}
  % \note{Exoskeleton robots developed by some universities and companies}
\end{figure}

智能轮椅不仅要能感知世界、表达所学知识、做出有用的推断和计划,还必须能够与其他智能体,特别是与人进行有效的交流。由于缺乏运动能力、缺乏力量或视觉障碍,大量的使用者很难独立操作轮椅\cite{hartmanHumanMachineInterfaceSmart2019},因此在操控智能方面主要的工作是各种替代性的人机交互接口,以增加用户的自主权。对于患有运动神经元疾病(如肌萎缩侧索硬化症)的人群,脑机接口(BCI)是目前最有应用潜力的研究方向之一。脑电图(EEG)是BCI中使用的主要非侵入性技术之一,在图\ref{fig:1-3}(A)中给出了BCI操控电动轮椅的主要技术路线\cite{naserPracticalBCIDrivenWheelchairs2023}。其中,P300和SSVEP是两种最常用用于识别用户的意图的BCI范式。P300范式依赖于检测用户大脑中的P300电位\cite{rebsamenBrainControlledWheelchair2010},而SSVEP范式依赖于检测用户大脑中的视觉诱发电位\cite{dongMultimodalBrainComputer2022,ngIndirectControlAutonomous2020a}。然而,基于EEG信号的人机界面通常存在带宽较低、学习周期长、计算要求高以及需要用户高度集中注意力等问题。此外,它们大多只能在有限的离散指令空间内工作。最近,有用户研究显示,瘫痪患者普遍表示更喜欢将人机界面嵌入可穿戴设备中\cite{zhangUnderstandingInteractionsSmart2022}。围绕这一问题,美国西北大学的Thorp等人\cite{thorpUpperBodyBasedPower2016d}针对四肢瘫痪患者设计了一种基于身体-机器接口(BMI)的轮椅控制界面,通过放置在肩部的几个反光标记点以及光学相机,该系统可以允许使用者通过高维肩关节运动变化生成轮椅控制命令,进而提高用户操作的灵活性。随后Seáñez-González等人在此基础进一步设计了一套基于惯性传感器网络的体-机交互设备,并深入地分析了不同数据映射方式在虚拟电动轮椅操控效果上的表现。此外,在反馈交互增强方面,Devigne等人\cite{devigneDesignHapticGuidance2018}设计了一套低复杂度优化框架的触觉反馈电动轮椅导航解决方案,通过向轮椅操纵杆发送力反馈来提供导航避障信息,以帮助轮椅用户更好地理解周围环境。Schettino等人\cite{schettinoImprovingGeneralisationLearning2020}基于一个力反馈控制器提出了一种学习辅助驾驶(LAD)策略,通过使用自动编码器和高斯过程等模型来学习示教辅助策略以帮助轮椅使用者在空间中更好移动。
\begin{figure}[h]
  \centering
  \includegraphics[width=1\textwidth]{1-fig-3.pdf}
  \caption{智能轮椅操控界面相关研究成果}
  \label{fig:1-3}
  % \note{Exoskeleton robots developed by some universities and companies}
\end{figure}

\subsection{站立辅助机器人}
人类由坐到站的转移(STS)是一种经常进行的日常活动,其中由于肌肉无力、关节病变、神经系统问题以及骨密度降低导致无法完成正常STS转移在高龄老人中普遍存在,这一问题对老年人的生活自理能力和生活质量产生了显著影响。除了为截瘫患者开发的全下肢外骨骼系统外,当前仅为支持人体由坐到站转移而专门设计的辅助机器人设备和相关研究并不是很多,这些设备和被辅助的对象的交互控制机制可划分为三个主要类别,即位置运动控制、力控制和开关控制。

\section{人-机器人交互界面研究现状}
人机交互目前是一个非常广泛和多样化的研究领域,致力于理解、设计和评估机器人系统,供人类使用或与人类进行物理接触或远程操控。HRI的问题本质上是通过评估双方的能力和设计形成合适互动的技术来理解和塑造人类和机器人之间的互动。这构成了一个需要结合多学科知识的研究领域,如神经科学、计算机科学、心理学、认知科学、医学以及工业设计等\cite{mohebbiHumanRobotInteractionRehabilitation2020a}。
\section{人机共享控制研究现状}

\section{研究现状总结及其存在的关键问题}
大多数辅助和康复机器人的设计在功能上能够提供所需的输出和执行预期任务,但由于没有使用最先进的技术,也没有在设计和控制概念中充分考虑人类用户的需求,因此效率受到质疑。因此,在设计中需要一种以人为中心的方法,考虑各个方面,如人类神经和肌肉骨骼系统,主要需要关注人类机器人交互(HRI)和交互界面技术

\section{研究内容及章节结构安排}

\subsection{二级节标题}

\subsubsection{三级节标题}

\paragraph{四级节标题}

\subparagraph{五级节标题}

本模板 \pkg{ustcthesis} 是中国科学技术大学本科生和研究生学位论文的 \LaTeX{}
模板, 按照《\href{https://gradschool.ustc.edu.cn/static/upload/article/picture/ce3b02e5f0274c90b9331ef50ae1ac26.pdf}
{中国科学技术大学研究生学位论文撰写手册}》(以下简称《撰写手册》)和
《\href{https://www.teach.ustc.edu.cn/?attachment_id=13867}
{中国科学技术大学本科毕业论文(设计)格式}》的要求编写。

Lorem ipsum dolor sit amet, consectetur adipiscing elit, sed do eiusmod tempor
incididunt ut labore et dolore magna aliqua.
Ut enim ad minim veniam, quis nostrud exercitation ullamco laboris nisi ut
aliquip ex ea commodo consequat.
Duis aute irure dolor in reprehenderit in voluptate velit esse cillum dolore eu
fugiat nulla pariatur.
Excepteur sint occaecat cupidatat non proident, sunt in culpa qui officia
deserunt mollit anim id est laborum.



\section{脚注}

Lorem ipsum dolor sit amet, consectetur adipiscing elit, sed do eiusmod tempor
incididunt ut labore et dolore magna aliqua.
\footnote{Ut enim ad minim veniam, quis nostrud exercitation ullamco laboris
  nisi ut aliquip ex ea commodo consequat.
  Duis aute irure dolor in reprehenderit in voluptate velit esse cillum dolore
  eu fugiat nulla pariatur.}
