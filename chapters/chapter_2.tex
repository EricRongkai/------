\chapter{研究理论基础}

\section{闭环人机交互过程的不确定性分析}
人-机器人交互问题可以被认为是一个协商性的协同控制过程,在此过程中使用者与系统共同沟通意图。因此,每一种类型的交互过程都有某种形式的不确定性,通常我们的目标都是引入更多的传感器信息去试图设计一个更精准的方法去推断使用者的意图从而忽略其中不确定因素的存在。我们认为在人机交互过程中的不确定性本身是一种信息,应该需要被正确的认识和对待,并且必须表示和处理所存在的不确定性,而不是盲目地过滤或忽略掉它。人机交互过程的不确定性来源繁杂,涉及人的行为、决策、机器的感知和决策、协同任务的目标设定以及交互方式等多个方面,大致有以下几种类型:(1)人的行为不确定性:人的行为受到情绪、健康状况、认知能力等因素的影响,导致其在与机器进行协同工作时的表现不确定。(2)人类决策的不确定性:人类在面临选择时,由于信息不完全、知识有限、环境复杂等因素导致无法确定最佳选择的情况。(3)机器的感知不确定性:机器在与人进行协同工作时,需要对人的行为和环境进行感知和理解。然而,由于感知技术的限制,机器对于人的行为和环境的理解往往存在不确定性。(3)机器决策的不确定性:由于机器学习算法的限制或数据的不完整性、不准确性等因素导致机器无法确定最佳决策的情况。(4)协同任务的目标不确定性:协同任务的目标通常由人和机器共同制定,但由于人和机器的意见、知识和经验的不同,目标的制定存在一定的不确定性。(5)交互方式的不确定性:人机协同中的交互方式往往是通过交互界面、语言或手势等方式进行的,但这些交互方式存在一定的不确定性。

已有研究对人-机器人交互在多智能体框架中进行建模分析,其中机器人和人被都被视为智能代理\cite{liuDesigningRobotBehavior}。因此,我们首先从具有较好理论研究基础的智能机器人行为设计出发,对机器智能的设计实现过程进行分析。为了尽可能地适用于更多的交互场景和传感器使用,在此基础上,通过将人类交互行为视为一个连续的闭环控制过程,我们进一步分析了交互过程中的不确定性因素来源。

\subsection{机器人行为建模}



\section{人机共享控制}

\section{机器人轨迹模仿学习}