\chapter{总结与展望}
现阶段关于康复辅助机器的智能化研究正在逐渐从``以机器为中心''向``以人为中心''过渡。相较于传统的机器人研究工作一般在功能上满足使用者的需求,``以人为中心''的机器人研究需要进一步从安全性、遵从性、舒适度、个体适应性以及灵活性等多方面的性能指标综合考虑。其中,在人机交互系统的设计研究中需要综合考虑安全性和个体适应性是康复辅助机器人区别于其他种类机器人的主要特征。然而这两者往往是相悖的,过分强调安全排除一切可能造成风险的不确定性因素会导致机器人倾向于保守进而降低使用者体验。另一方面,人类行为虽然在许多方面明显有规则约束并且可预测,但它也始终是可变的、个体的,甚至是随机的,将更多使用者的自主行为引入交互系统中则会导致系统存在不可控的风险。人机交互过程中的不确定性本身是一种信息,应该需要被正确地认识和对待,并且必须表示和处理所存在的不确定性,而不是盲目地过滤或忽略掉它。通过有效地应对人机交互中各个环节的不确定性,康复辅助机器人能够根据患者的需求和反馈进行个性化调整,从而实现康复辅助机器人在个体适应性和安全性之间的最佳平衡。本文的主要贡献和创新总结如下:

\begin{table}[htb]
        \centering
        \caption{本文的主要研究对象以及采用的技术方式总结对比}
        \begin{tabular}{p{2cm}p{2cm}p{2cm}p{3.2cm}p{3.2cm}}
        \toprule
                应用领域& 人类交互动作来源\&交互感知方式 & 交互数据类型\&意图推断处理方式 & 人机交互不确定性分析 & 共享自主系统实现目的 \\ 
        \midrule
                移动辅助机器人体感操控 & 肩部肩胛骨随机运动\&基于可穿戴柔性应变传感器与惯性传感器 & 高维度连续\&基于状态估计的连续随机意图推理 &将意图模型转移不确定性和观测不确定性建模为多维高斯分布,利用了卡尔曼滤波器进行最优状态估计 & 基于先验操控数据,实现交互意图的自适应介入,用于提高交互效率\\ 
    
                坐立辅助机器人人体下肢关节运动感知 & 下肢原地点到点坐立运动\&基于穿戴标记点的光学运动捕捉系统 & 三维连续 \&基于随机系统模型参数辨识的意图推理	&将意图模型转移不确定性和观测不确定性建模为多维高斯分布,基于期望最大化算法从累积的观测数据中推断出轨迹编码模型中关于轨迹持续时长的系统参数 & 基于意图推理结果置信度表征,实现机器人辅助轨迹优化计算的意图推理集成,用于提高个体适应性\\ 
    
                关节运动辅助机器人人体膝关节运动感知 & 移动中下肢步态连续周期性运动\&基于可穿戴惯性传感器 & 单维连续\&基于时间序列压缩编码的意图推理	&基于多维高斯模型,在编码后的向量空间中对两种步行交互动作不确定性分布进行了建模。通过概率计算,实现了连续运动下的步态模式切换交互意图推理和识别。& 基于意图推理和动作概率先验分布,验证和纠正用户输入偏差,平衡自学习式康复辅助轨迹生成的个体适应性和安全性\\ 
        \bottomrule
        \end{tabular}
        \label{tab:6-1} 
      \end{table}   
\begin{enumerate}
\item 从机器人行为系统设计出发并基于闭环控制的思想,本文首先对人机交互过程中由人类导致的交互不确定性来源进行了分析,并主要围绕人类动作的不确定性展开了应用研究。康复辅助机器人中最为广泛的需求是对人类的物理动作的感知,而人类动作的不确定性是由感知、认知和执行等环节共同决定的,采用了各类型传感器的人机交互接口是对人类的物理动作的量化表征。通过交互设备的感知数据实现更高层次的人类行为(意图)理解是处理人机交互问题的关键。因此我们认为,在``以人为中心''的人机交互研究中,解决问题的目标应当不仅仅是一个简单明确的识别或回归算法,而是需要一个可以自适应处理交互中不确定的新框架。

\item 针对交互中的不确定自适应问题,在研究中引入了共享自主方法,并详细地介绍了该方法在处理人机交互自适应问题中的优势。共享自主通过实时感知与分析环境,使得机器人能够更加适应与人之间的物理交互,并具备持续进化的特性。围绕该方法,本文深入探讨和研究了其在以下三个典型场景中的应用潜力与挑战:定制化人机交互接口在移动辅助机器人操控中的自适应交互指令解码、站立辅助机器人中被辅助对象的运动交互速度自适应感知,以及用于偏态步态康复的主动式膝关节辅助机器人自适应运动轨迹规划方法。并为此分别搭建了一套基于柔性拉伸/弯曲传感器的定制化人机交互系统、一套面向失能弱能人群的站立辅助的机器人运动轨迹优化框架以及一套用于偏瘫患者步态对称性康复训练的主动式膝关节矫形器系统用以支撑研究。

\item 针对上肢截肢或四肢瘫痪患者使用肩部运动操控外部辅助设备的需求,在第三章中基于所设计的新型可穿戴人机交互接口提出了一种自适应切换的交互接口数据解码方法,并进行了实验研究。它将用户的使用确定性数据映射模式操控光标的先验表现信息集成到一个共享自主系统的非线性仲裁函数中,实现确定性数据映射和意图推理两种数据解码方法的自适应切换。该方法的有效性通过一系列光标操控任务以及一个虚拟轮椅驾驶任务进行了验证。结果表明自适应解码方式切换不仅在光标操控中提高了使用者操控命令生成的准确性,同时也保证了数据解码方法的动态性能以适应轮椅操控等任务的要求。

\item 针对人类在完成坐立运动时的需要机器人适应人类运动速度的不确定性问题的需求,在第四章中提出了一种人类坐立运动时间意图的推断方法,并进行了实验验证。通过离线采集真实场景下的人体坐立离线运动数据,建立了一个由概率化离散动态运动基元表征的先验坐立运动技能库。通过将在部分观测下的运动时间估计看作一个系统参数辨识问题,基于期望最大化算法迭代优化坐立动作模型的时间缩放系数,实现了连续的运动时间估计。实验表明,所设计的站立动作速度估计方法通过一个估计置信度水平量化指标可以嵌入一个共享自主系统中,实现辅助机器人的在线运动轨迹的稳定优化或变刚度控制。

\item 针对偏瘫患者的步态对称性康复的需求,在第五章中基于所搭建的主动式膝关节矫形器平台提出了一种通过学习偏瘫健康侧步态特征的在线对称步态轨迹生成方法,并进行了实验验证。该方法通过融合节律型动态运动基元与一个自适应非线性频率振荡器,实现了在线的步态运动周期轨迹的编码与解码,同时实现了下肢健侧和患侧的步态相位自适应延迟。此外,通过对健康人群的离线步行任务采集数据,设计了一个带有先验步态技能库的共享自主模块,用以分析和仲裁来自健侧的实时用户输入,从而最大程度地减轻输入不确定性的影响。实验表明,所设计方法可以在学习使用者自身步态特征的同时,有效地降低由用户随机输入导致的安全风险。
\end{enumerate}

在表\ref{tab:6-1}中,我们对本文所介绍关于交互不确定性处理的各项研究的思路与采用的技术方法进行了总结。当交互动是随机的且无固定模式时,我们在研究中将意图推理看作一个随机系统中的最优状态估计问题,基于一个标准的卡尔曼滤波方法实现了连续的意图推理。进一步地,当交互动作轨迹存在一个固定的拓扑模式时,我们在研究中将意图推理作为一个系统参数辨识问题,实现了依靠运动轨迹不完整观测中的连续意图估计。最后,当交互动作存在固定拓扑模式且具有周期性质时,基于一个编码器-解码器结构的交互算法,实现了在低维向量子空间中的交互不确定性处理。综合来看,``以人为中心''的康复辅助机器人交互研究涉及多个领域的知识,目前相关理论研究尚不成熟并且仍处于探索阶段。针对不同的应用场景以及交互感知方式,实现对人类动作(意图)的理解并将其融入一个共享自主系统中仍然是特异的。在未来,随着康复辅助机器人的智能化研究从感知智能逐步迈向认知智能阶段,如何充分利用更丰富的感知信息以及更强大的计算资源设计共享自主系统,是实现通用辅助机器人在实际生活中落地应用的重要研究方向。